\begin{verbatim}
%% начальные условия------------change
T=InitialConditions.T;
K=InitialConditions.K;
h=InitialConditions.h;
d=InitialConditions.d;

syms p Psi ;
u=Psi ;
disp('        (T^2*p^2+2*h*T*p+1)+K*u ');
D=(T^2*p^3+2*h*T*p^2+p)+K*u ;
disp('           D/(T^2)');
D=collect(D,p);
D=D/(T^2);
D=collect(D,p);
vpa(D,3)
Dp=coeffs(D,p);

a=double([0 Dp(2) Dp(3)]);
b=double(Dp(1)/Psi)
e=[-1 2*a(3) -(a(3)^2+a(2)) a(2)*a(3) ]/b ;

% выбор праметров 
max_inter=10; %---количество итераций максимальное
mn=[2,0.5];%change множитель у alpha и  betta соотв.
near=0;
far=1; %------------------------начало интервалов
step=1; %-------------------------длинна интервала
iter=1;
IT=0;
n=2;
find=1;
tic;

while ((n>1)||(IT==0))
fprintf('\r%s',['итерация номер ',num2str(iter)]);
%выбор C1
near=far;
far=far+step;
%выбор случайного числа из интервала
inter=[near far];
C2=rand*(inter(2)-inter(1))+inter(1);
%---------------------------------
C1=C2^2-a(3)*C2+a(2);
if ~(sign(C1)+1)
iter=iter+1;
continue
end
f=e(1)*C2^3+e(2)*C2^2+e(3)*C2+e(4);
alpha=f+abs(mn(1)*f);
betta=f-abs(mn(2)*f);
%проверка устойчивости
psiv=(-C1*(C2-a(3)))/b;
D1=subs(D,Psi,psiv);
pv=double(solve(D1,p));
n=0;
for i=1:length(pv)
if real(pv(i))>0
n=n+1;
end
end
%---------------------------
%проверка попадания изображающей точки 
%на плоскость скольжения
psiv=alpha;
D1=subs(D,Psi,psiv);
pv=double(solve(D1,p));
IT=1;

for i=1:length(pv)
if pv(i)==real(pv(i))
if real(pv(i))>=0
IT=0;
end
end
end
%----------------------------
if iter>max_inter, find=0; break,else, find=1; end
iter=iter+1;
end

%непонятная вещь, которую необходимо сделать,
% чтобы получить скольжение
C2=C2*3;

\end{verbatim}