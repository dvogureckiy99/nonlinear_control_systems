\section{Исследование влияния широтно-импульсной модуляции на качество проектируемой системы}
Существенное влияние на свойства разрабатываемой системы управления могут ввести особенности технической реализации управляющего устройства. В частности, при высоком уровне энергии в управляемом объекте реализация управляющих воздействий в виде непрерывных сигналов практически невозможна в силу больших потерь в выходных каскадах исполнительных устройств. В таких случаях выходные каскады выполняют в виде импульсных устройств, в которых непрерывный сигнал квантуется по уровню или по времени, а затем преобразуется в импульсную модулированную последовательность. Одним из видов модуляции является широтно-импульсная (ШИМ), при которой непрерывный сигнал преобразуется в последовательность импульсов одинаковой амплитуды при постоянном периоде повторения Т, а длительность импульса определяется по какому-либо закону, например, линейному, в зависи¬мости от значения входного модулируемого непрерывного сигнала.

Управляющий сигнал формируется компаратором, когда на его инвертирующий вход подается пилообразный сигнал, а на неинвертирующий — модулирующий непрерывный сигнал. Выходные импульсы получаются прямоугольными, частота их следования равна частоте пилы, а длительность положительной части импульса связана с временем, в течение которого уровень модулирующего сигнала, подаваемого на неинвертирующий вход компаратора, оказывается выше уровня сигнала пилы, который подается на инвертирующий вход. Когда напряжение пилы выше модулирующего сигнала, на выходе будет отрицательная часть импульса.

Следует отметить, что при смене полярности модулируемого сигнала полярность импульсов также меняется. Длительность импульсов на выходе модулятора может меняться от нуля до непрерывного сигнала, при этом амплитуда и период повторения (частота) следования импульсов остаются неизменными.

Схема симулинк с добавлением ШИМ на  рис.\ref{fig:sim_final_VSS_PWM}. 
\begin{sidewaysfigure}[!h]\centering
	\includegraphics[width=1.0\linewidth]{images/sim_final_VSS_PWM}
	\caption{Структурная схема нелинейной СПС с ШИМ. }\label{fig:sim_final_VSS_PWM}
\end{sidewaysfigure}

%Рассмотрим влияние параметров схемы на ПП системы.
%Начнем с коэффициентов $K_1,K_2$, они работают при отклонении системы "в большом".
%Переходные процессы для разных значений этих параметров на рис.\ref{fig:final_VSS_PWM_k2},графики на всем интервале времени практически не изменяются, только в конце ПП при переключении систем управления появляются различия. Для более детального рассмотрения поакзана приближенная картинка на рис. \ref{fig:final_VSS_PWM_k2_zoom}.
%Они представлят собой коэффициенты пропорционального и дифференциального регулятора соответственно.
%Соответственно влияют они на систему также, как и коэффициенты ПД-регулятора в пункте \ref{title:PDR}: $k_1$ --- коэффициент пропорционального звена, влияет на скорость нарастания выходной величины, а $k_2$ --- коэффициент дифференцирующего звена, влияет на колебательность системы.
%При этом при разных сочетаниях коэффициентов время ПП практически не улучшается и наилучшее время составляет $133.46$ сек.
%\begin{figure}[!h]\centering
%	\includegraphics[width=1\linewidth]{images/final_VSS_PWM_k2}
%	\caption{ Графики ПП при вариации параметров $k_1,k_2$ в <<большом>>.}\label{fig:final_VSS_PWM_k2}
%\end{figure}
%\begin{figure}[!h]\centering
%	\includegraphics[width=1\linewidth]{images/final_VSS_PWM_k2_zoom}
%	\caption{ Приближенный график ПП при вариации параметров $k_1,k_2$.}\label{fig:final_VSS_PWM_k2_zoom}
%\end{figure}
%
%Теперь рассмотрим коэффициенты, влияющие на ПП "в малом": $C1,C2,a,b$. Также стоит отметить порог, отвечающий за переключение систем управления (переключение структур) при превышении абсолютными значениями ошибки данного порога. Данный порог определяет в какой момент переключить сруткуру, дргуими словами, он определяет при какие отклонения можно считать "большими", а какие "малыми". Так как при настройке системы отклонение "в малом" устанавливалось равным $1$, то и знаечние порога равно $1$. 
%Параметры $a$ и $b$ незначительно влияют на ПП ситемы при их отклонениях при правильной настройке.
%Это можно увидеть на рисунке \ref{fig:final_VSS_PWM_ab}.
%\begin{figure}[!h]\centering
%	\includegraphics[width=1\linewidth]{images/final_VSS_PWM_ab}
%	\caption{ Графики ПП при вариации параметров $a,b$ в <<малом>>.}\label{fig:final_VSS_PWM_ab}
%\end{figure}
%Параметры $C1$ и $C2$ на колебательность и скорость нарастания. При увеличении $C1$ колебательность и скорость нарастания возрастает, а при увеличении $C2$ колебательность уменьшается. На рис. \ref{fig:final_VSS_PWM_c1c2} ПП ,  соответствующий техническим требованиям, при разных комбинациях параметров (На рис.\ref{fig:final_VSS_PWM_c1c2_zoom} в увеличенном масштабе). 
%\begin{figure}[!h]\centering
%	\includegraphics[width=1\linewidth]{images/final_VSS_PWM_c1}
%	\caption{ Графики ПП при вариации параметров $C1,C2$ в <<малом>>.}\label{fig:final_VSS_PWM_c1c2}
%\end{figure}
%\begin{figure}[!h]\centering
%	\includegraphics[width=1\linewidth]{images/final_VSS_PWM_c1_zoom}
%	\caption{ Графики ПП при вариации параметров $C1,C2$ в <<малом>>.}\label{fig:final_VSS_PWM_c1c2_zoom}
%\end{figure}
%
%При переходе к системе с ШИМ появляется постоянная ошибка и отрицательный выброс в противоположом с перерегулированием направлении
%, теперь коэффициент $a$ начинает играть решающее значение в виде ПП. Чем он ниже, тем больше отрицательный выброс. Также при увеличении коэфф. $a$ установившейся сигнал поднимается вше, тем самым уменьшается ошибка регулирования. На рис.\ref{fig:final_VSS_PWM_a} графики ПП при вариации параметров $a$ в <<малом>> в схеме с ШИМ(На рис.\ref{fig:final_VSS_PWM_a_zoom} в увеличенном масштабе).
При переходе к системе с ШИМ появляется постоянная ошибка
, теперь коэффициент $a$ начинает играть решающее значение в виде ПП. При увеличении коэфф. $a$ установившейся сигнал поднимается вше, тем самым уменьшается ошибка регулирования. На рис.\ref{fig:final_VSS_PWM_a} графики ПП при вариации параметров $a$ в <<малом>> в схеме с ШИМ(На рис.\ref{fig:final_VSS_PWM_a_zoom} в увеличенном масштабе).
\begin{figure}[!h]\centering
	\includegraphics[width=1\linewidth]{images/final_VSS_PWM_a}
	\caption{ Графики ПП при вариации параметров $a$ в <<малом>> в схеме с ШИМ.}\label{fig:final_VSS_PWM_a}
\end{figure}
\begin{figure}[!h]\centering
	\includegraphics[width=1\linewidth]{images/final_VSS_PWM_a_zoom}
	\caption{ Увеличенные графики ПП при вариации параметров $a$ в <<малом>> в схеме с ШИМ.}\label{fig:final_VSS_PWM_a_zoom}
\end{figure}

Период квантования влияет на колебания системы в установвившемся состоянии, при его увеличении колебания уменьшаются.
Путем последовательных 
приближений было определено значение периода квантования $T{\text{кр}}$, при 
котором устойчивая без модулятора система становится неустойчивой, оно равно 4 сек.
На рис.\ref{fig:final_VSS_PWM_Ts45_gran} изображены переходный процессы при значениях, близких к $T{\text{кр}}$.

\begin{figure}[!h]\centering
	\includegraphics[width=1\linewidth]{images/final_VSS_PWM_Ts45_gran}
	\caption{ График переходного процесса при близких к $T_{\text{кр}}$ значениях.}\label{fig:final_VSS_PWM_Ts45_gran}
\end{figure}

Периода квантования $T_{\text{шим}}$, при котором система будет обладать теми же качествами, что и 
исходная равен 0,01 сек. Тогда соответствующий ему шаг решения равен 0,0001.  

Исследуем движение фазовых координат во времени посредством моделирования процессов в системе при отклонении системы от состояния равновесия. 
На рис.\ref{fig:final_VSS_PWM_sv_DCS_bol},\ref{fig:final_VSS_PWM_sv_DCS_mal} указано изменение выходной переменной. 
\begin{figure}[!h]\centering
	\includegraphics[width=1\linewidth]{images/final_VSS_PWM_sv_DCS_mal}
	\caption{ Графики изменения переменных состояния в <<малом>>.}\label{fig:final_VSS_PWM_sv_DCS_mal}
\end{figure}
\begin{figure}[!h]\centering
	\includegraphics[width=1\linewidth]{images/final_VSS_PWM_sv_DCS_bol}
	\caption{ Графики изменения переменных состояния в <<большом>>.}\label{fig:final_VSS_PWM_sv_DCS_bol}
\end{figure}
\begin{figure}[!h]\centering
	\includegraphics[width=0.6\linewidth]{images/final_VSS_PWM_DCS_SAW}
	\caption{ Генерируем пилу}\label{fig:final_VSS_PWM_DCS_SAW}
\end{figure}
Время переходного процесса при отклонении "в малом" согласно рисунку рис.\ref{fig:final_VSS_PWM_sv_DCS_mal} равно 14.42 сек.
Время переходного процесса при отклонении "в большом" согласно рисунку рис.\ref{fig:final_VSS_PWM_sv_DCS_bol} равно 167 сек.
В результате видим, что переходный процесс по прежнему удовлетворяет условиям задания.
\begin{figure}[!h]\centering
	\includegraphics[width=0.6\linewidth]{images/final_VSS_PWM_DCS_PWM}
	\caption{ Воздействие на объект (ШИМ)}\label{fig:final_VSS_PWM_DCS_PWM}
\end{figure}
\begin{figure}[!h]\centering
	\includegraphics[width=0.6\linewidth]{images/final_VSS_PWM_DCS_upr}
	\caption{ управляющий непрерывный сигнал }\label{fig:final_VSS_PWM_DCS_upr}
\end{figure}
\begin{figure}[!h]\centering
	\includegraphics[width=0.6\linewidth]{images/final_VSS_PWM_DCS_upr_ogr}
	\caption{ управляющий непрерывный сигнал после насыщения}\label{fig:final_VSS_PWM_DCS_upr_ogr}
\end{figure}
\begin{figure}[!h]\centering
	\includegraphics[width=0.6\linewidth]{images/final_VSS_PWM_DCS_upr_ogr_dig}
	\caption{ управляющий дискретизированный сигнал }\label{fig:final_VSS_PWM_DCS_upr_ogr_dig}
\end{figure}


