\anonsection{Введение}
	В настоящее время успешное решение задач автоматизации в настоящее время тесно связано с использованием современных технологий, теоретических и практических разработок автоматических систем управления.
	
Синтез структуры и определение параметров управляющего устройства выполняются в определенной последовательности по классической схеме. Вначале исследуются свойства управляемого объекта по его характеристикам. Структура и параметры объекта при этом считаются известными, процессы в объекте описываются нелинейными дифференциальными уравнениями. Нелинейность объекта вызвана нелинейностью одного из устройств с типовой нелинейной характеристикой. Определение требуемой структуры управляющего устройства ведется путем итераций от самых простых решений до сложных структур на основе требований технического задания. Вначале исследуются возможности системы регулирования с пропорциональным и пропорционально - дифференциальным регуляторами. Показывается, какие требования технического задания могут быть удовлетворены с помощью указанных структур управляющего устройства. Известно, что с помощью линейных регуляторов можно получить требуемое качество переходной характеристики по форме, однако при этом можно потерять другие качества системы, такое как быстродействие.  Из теории линейных систем известно, что монотонный характер переходной характеристики  можно получить при плохом быстродействии системы управления. В тоже время при применении нелинейных структур управляющих устройств, в частности, систем с переменной структурой,  эти противоречия можно успешно разрешить. В связи с этим обстоятельством основное внимание в работе уделено вопросам синтеза одной из наиболее распространенных разновидностей систем с переменной структурой,  а именно системы со скользящим режимом движения.

При синтезе систем с переменной структурой со скользящим режимом движения решаются основные теоретические вопросы – выясняются условия существования скользящих режимов, условия устойчивости скользящих режимов  при  больших и малых начальных отклонениях. Существенные коррективы в структуру управляющего устройства системы вносит наличие в составе системы нелинейного элемента. В  качестве примера в работе рассмотрен синтез системы с переменной структурой при наличии в системе нелинейного элемента  вида насыщение, это обстоятельство значительно усложняет теоретические разработки, так как при больших начальных отклонениях  система управления становится релейной. Исследование релейных систем приводится для любых видов релейных  характеристик, так как в разрабатываемой системе могут быть нелинейности с разнообразными характеристиками.
