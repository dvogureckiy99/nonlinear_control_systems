\documentclass[a4paper,fontsize=14bp]{extarticle}

\linespread{1.3} % полуторный интервал
\usepackage{extsizes} %добавляет в параметры класса документа 14pt
\usepackage{scrextend} % дополняет другие классы частью возможностей KOMA-Script
\usepackage[left=3cm, right=2cm, top=2cm, bottom=2cm]{geometry}% должен обязательно стоять после extdize
%для установки шрифтов % \fontsize{14bp}{\baselineskip}\selectfont


%\linespread{1.3} % полуторный интервал
%\renewcommand{\rmdefault}{ftm} % Times New Roman
%\frenchspacing

%%% Работа с русским языком
\usepackage[english,russian]{babel}   %% загружает пакет многоязыковой вёрстки
\usepackage{fontspec}      %% подготавливает загрузку шрифтов Open Type, True Type и др.
\defaultfontfeatures{Ligatures={TeX},Renderer=Basic}  %% свойства шрифтов по умолчанию
\setmainfont[Ligatures={TeX,Historic}]{Times New Roman} %% задаёт основной шрифт документа
\setsansfont{Calibri}                    %% задаёт шрифт без засечек
\setmonofont{Courier New}
\usepackage{indentfirst}
\frenchspacing


%%% Дополнительная работа с математикой
\usepackage{amsfonts,amssymb,amsthm,mathtools} % AMS
\usepackage{amsmath}
\usepackage{icomma} % "Умная" запятая: $0,2$ --- число, $0, 2$ --- перечисление

%%% Номера формул
\mathtoolsset{showonlyrefs=true} % Показывать номера только у тех формул, на которые есть \eqref{} в тексте.

%% Перенос знаков в формулах (по Львовскому)
\newcommand*{\hm}[1]{#1\nobreak\discretionary{}
	{\hbox{$\mathsurround=0pt #1$}}{}}

%%% Работа с картинками
\usepackage{graphicx}  % Для вставки рисунков
\graphicspath{{images/}}  % папки с картинками
\setlength\fboxsep{3pt} % Отступ рамки \fbox{} от рисунка
\setlength\fboxrule{1pt} % Толщина линий рамки \fbox{}
\usepackage{wrapfig} % Обтекание рисунков и таблиц текстом
%\usepackage[process=all,crop=pdfcrop]{pstool}%для картинок

%%% Работа с таблицами
\usepackage{array,tabularx,tabulary,booktabs} % Дополнительная работа с таблицами
\usepackage{longtable}  % Длинные таблицы
\usepackage{multirow} % Слияние строк в таблице

%%Подписи под изображениями и таблицами
\usepackage[tableposition=top]{caption}
\usepackage{subcaption}
\DeclareCaptionLabelFormat{gostfigure}{Рисунок #2}
\DeclareCaptionLabelFormat{gosttable}{Таблица #2}
\DeclareCaptionLabelSeparator{gost}{~---~}
\captionsetup{justification=centering}
\captionsetup{labelsep=gost}
\captionsetup[figure]{labelformat=gostfigure}
\captionsetup[table]{labelformat=gosttable}
\renewcommand{\thesubfigure}{\asbuk{subfigure}}

%Нумерация страниц
\usepackage{fancyhdr}
\pagestyle{fancy}
\fancyhf{}
\fancyfoot[C]{\thepage}
\fancyheadoffset{0mm}
\fancyfootoffset{0mm}
\setlength{\headheight}{14pt}
\renewcommand{\headrulewidth}{0pt}
\renewcommand{\footrulewidth}{0pt}
\fancypagestyle{plain}{ 
	\fancyhf{}
	\rhead{\thepage}}
\setcounter{page}{5} % начать нумерацию страниц с №5

%%------------------заголовки ------------

\usepackage{titlesec}

\titleformat{\chapter}[display]
{\filcenter}
{\MakeUppercase{\chaptertitlename} \thechapter}
{8pt}
{\bfseries}{}

\titleformat{\section}
{\normalsize\bfseries\centering\MakeUppercase}
%{\filcenter}
{\thesection}
{1em}{}

%\setcounter{section}{0} %если хотим убрать номера
%\newcommand{\ssection}[1]{\section*{#1}\addtocounter{section}{1}}

\titleformat{\subsection}
{\normalsize\bfseries\centering}
{\thesubsection}
{1em}{}

\titleformat{\subsubsection}
{\normalsize\bfseries\centering}
{\thesubsubsection}
{1em}{}

% Настройка вертикальных и горизонтальных отступов
\titlespacing*{\chapter}{0pt}{-30pt}{8pt}
\titlespacing*{\section}{\parindent}{*4}{*4}
\titlespacing*{\subsection}{\parindent}{*4}{*4}

\newcommand{\anonsection}[1]{\section*{#1}\addcontentsline{toc}{section}{#1}}

%для поворота рисунка
\usepackage{rotating}

