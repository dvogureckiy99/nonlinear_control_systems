%--------------------------HOLE-------------------------------
% Что это?
% Это файл с простым текстом( выводы, описание чего-либо), который редактируются вручную.
% Описание (для чего файл):
% Hole number:2

\subsection{выполнение работы}
\subsubsection{Система с переменной структурой} \label{sub1}
Создали новую модель в Matlab Simulink на рис.\ref{fig:sim_variable_structure1}.
Коэффициенты были подобраны таким образом, что $k_1>k_2$. Коэффициенты указаны в табл.
\ref{tab:tab2}.

Математическая форма записи описанного алгоритма управления примет вид системы \eqref{eq:eq3}.
\begin{equation} \label{eq:eq3}
\left\{ 
	\begin{aligned}
\text{\"{x}}+k_1\,k\,\text{x}=0&,\text{x\.{x}}>0, \\
\text{\"{x}}+k_2\,k\,\text{x}=0&,\text{x\.{x}}<0.
	\end{aligned}
\right.
\end{equation}

Система с переменной структурой  переключается с одного регулятора на другой в зависимости от  выполнения условий.

\begin{wraptable}{r}{0.5\linewidth}
	\caption{ Таблица коэффициентов.} \label{tab:tab2}
	\begin{tabular}{|c|c|c|c|c|}
		\hline
		номер & $k_1$ & $k_2$ & $x_0$ & $y_0$  \\ \hline
		1&\multirow{3}{*}{3} &  \multirow{3}{*}{0.8} & 0 & 0 \\ \cline{1-1} \cline{4-5} 
		2& & & 0.1 & 0.1 \\ \cline{1-1} \cline{4-5}
		3& & & 0.2 & 0.2 \\ \hline
	\end{tabular}
\end{wraptable}
Исследуем движение фазовых координат во времени посредством
моделирования процессов в системе при отклонении системы от состояния
равновесия. Значения начальных условий в табл.\ref{tab:tab2}. Фазовые траектории и переходные процессы в системе на рис.\ref{fig:variable_structure1}, крестиками указаны состояния системы, соответствующие начальным значениям. Как видно, система асимптотически устойчива, так как изображающая точка на фазовой траектории приближается к точке равновесия.

 В дополнение на рис.\ref{fig:variable_structure1_sys} указано изменение переменных состояния.
Отметим, что закон изменения $x$ представляет собой колебательный процесс, соответственно ─ $y$ тоже.


